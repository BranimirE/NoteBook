%% LyX 2.0.3 created this file.  For more info, see http://www.lyx.org/.
%% Do not edit unless you really know what you are doing.
\documentclass[oneside,spanish]{book}
\usepackage[T1]{fontenc}
\usepackage[latin9]{inputenc}
\usepackage[a4paper]{geometry}
\geometry{verbose,tmargin=3cm,bmargin=3cm,lmargin=3cm,rmargin=3cm}
\setcounter{secnumdepth}{3}
\setcounter{tocdepth}{3}

\makeatletter
%%%%%%%%%%%%%%%%%%%%%%%%%%%%%% User specified LaTeX commands.
\usepackage[T1]{fontenc}
\usepackage[latin9]{inputenc}
\usepackage{listings}
\usepackage[spanish]{babel}
\usepackage{times}

\usepackage{color}
\definecolor{javared}{rgb}{0.6,0,0} % cadenas
\definecolor{javagreen}{rgb}{0.25,0.5,0.35} % comentarios
\definecolor{javapurple}{rgb}{0.5,0,0.35} % palabras clave
\definecolor{javadocblue}{rgb}{0.25,0.35,0.75} % javadoc :D

\lstset{language=C++,
basicstyle=\ttfamily,
frame=single,
keywordstyle=\color{javapurple}\bfseries,
stringstyle=\color{javared},
commentstyle=\color{javagreen},
morecomment=[s][\color{javadocblue}]{/**}{*/},
numbers=left,
numberstyle=\small\color{black},
numbersep=10pt,
tabsize=4,
showspaces=false,
showstringspaces=false,
breaklines=true}

\makeatother

\usepackage{babel}
\addto\shorthandsspanish{\spanishdeactivate{~<>}}

\begin{document}

\title{NoteBook}

\maketitle
\tableofcontents{}


\chapter{AD-HOC}


\section{Utilidades}

\lstinputlisting{adhoc/utilidades.cpp}


\section{A�o bisiesto}

\lstinputlisting{adhoc/bisiesto.cpp}


\section{Doomsday}

\lstinputlisting{adhoc/doomsday.cpp}


\chapter{Estructuras}


\section{Union - Find}

\lstinputlisting{estructuras/unionfind.cpp}


\section{BIT}

\lstinputlisting{estructuras/bit.cpp}


\section{Segment Tree}

\lstinputlisting{estructuras/SegmenTree.cpp}


\section{Segment Tree con Lazy Propagati�n}

\lstinputlisting{estructuras/SegmenTree_lazy.cpp}


\section{Trie}

\lstinputlisting{estructuras/trie.cpp}


\chapter{Complete Search }


\chapter{Divide y venceras}


\section{MergeSort}

Primera implementacion:

\lstinputlisting{divideyvenceras/mergesort.cpp}

Segunda implementaci�n:

\lstinputlisting{divideyvenceras/mergesort2.cpp}


\chapter{Programaci�n Dinamica}


\section{El problema de la mochila}

\lstinputlisting{programaciondinamica/mochila.cpp}


\section{Cambio de modenas}

\lstinputlisting{programaciondinamica/coin_change.cpp}


\section{Distancia de edici�n (Algoritmo de Levenshtein)}

\lstinputlisting{programaciondinamica/levenshtein.cpp}


\chapter{Grafos}


\section{BFS}

\lstinputlisting{grafos/bfs.cpp}


\section{DFS}

\lstinputlisting{grafos/dfs_ff.cpp}


\section{Topological Sort}

G asiclico dirigido (DAG) 

Las tareas no son independientes y la ejecuci�n de una tarea s�lo
es posible si otras tareas que ya se han ejecutado. Solo hay un orden 

\lstinputlisting{grafos/TopSort.cpp}


\section{Dijkstra}

\lstinputlisting{grafos/dijstra.cpp}


\section{BellmandFord}

Si un grafo contiene un ciclo de coste total negativo entonces este
grafo no tiene soluci�n

\lstinputlisting{grafos/bellmandford.cpp}


\section{Componentes Fuertemente Conexas }

\lstinputlisting{grafos/scc.cpp}


\section{Componentes Fuertemente Conexas (Kosaraju)}

\lstinputlisting{grafos/Kosaraju.cpp}


\section{Minimum Spanning Tree (Kruskall)}

\lstinputlisting{grafos/kruskall.cpp}


\section{MaxFlow}

\lstinputlisting{grafos/MaxFlow.cpp}


\chapter{Matem�ticas}


\section{GCD}

\lstinputlisting{matematicas/gcd.cpp}


\section{LCM}

\lstinputlisting{matematicas/lcm.cpp}


\section{Exponenciaci�n rapida}

\lstinputlisting{matematicas/exponenciacion.cpp}


\section{Criba de Erathostenes}

\lstinputlisting{matematicas/criba.cpp}


\section{Triangulo de Pascal}


\section{Combinaciones(Para numeros muy grandes)}

\lstinputlisting{matematicas/combinaciones.cpp}


\section{Polinomios}

\lstinputlisting{matematicas/Polinomio.java}


\section{Fibonacci ( $O(log(n))$ )}

\lstinputlisting{matematicas/fibonacci.cpp}


\section{Multiplicaci�n entero por cadena}

\lstinputlisting{matematicas/multiplicacion.cpp}


\section{Multiplicaci�n de numeros grandes (Karatsuba)}

\lstinputlisting{matematicas/karatsuba.java}


\chapter{Cadenas}


\section{Utilidades}

\lstinputlisting{cadenas/utilidades.cpp}


\section{Boyer Moore}

Encuentra todos los match de un patron en el texto.

\lstinputlisting{cadenas/boyerMoore.cpp}


\section{KMP}

\lstinputlisting{cadenas/kmp.cpp}


\section{Iesima permutaci�n}

\lstinputlisting{cadenas/ipermutacion.cpp}


\section{Trie}

\lstinputlisting{cadenas/trie.cpp}


\chapter{Geometria Computacional}


\section{Intersecci�n de rectangulos}

\lstinputlisting{geometria/interseccionrect.cpp}
\end{document}
